\chapter{心得体会}
\begin{itemize}
    \item 数据库设计初期的需求分析和概念设计十分重要。在开始后期工作后,要想在修改数据库的定义将会是十分麻烦的。
    \begin{itemize}
        \item 由于早期的疏忽,我们对数据库中的某些属性考虑不周全(例如论文与评价的联系集属性、作者位次属性等),这导致我们需要中途修订表结构,进而导致一些已经完成的代码需要做级联的修改。
    \end{itemize}
    \item 对于数据库应用开发的了解不多,在摸索中学习。
    \begin{itemize}
        \item 尝试了前后端分离的开发方式。理解前后端工作原理。
        \item 学习使用flask后端框架和vue前端框架
    \end{itemize}
    \item 作为一个数据库实践作业,我们在实现的过程中认为数据库侧的相关功能实现并不是主要难点。难点在于如何将数据库应用到实际,并开发出一个可用的应用,其中最主要的困难在于前端vue代码的编写。
\end{itemize}
