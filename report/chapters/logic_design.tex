\chapter{数据库逻辑结构设计}

\section{关系模式设计}

根据概念结构设计得到的E-R图和转换规则,得到如下关系模式:
\begin{enumerate}
    \item 论文(论文编号,论文标题,论文链接,摘要,发表日期)
    \item 作者(作者编号,姓名,机构)
    \item 代码(代码编号,论文编号,代码链接,收藏数,框架)。外键:论文编号
    \item 方法(方法编号,方法名,方法描述)
    \item 评价指标(指标编号,得分计算方式,数据集编号,任务编号)。外键:数据集编号,任务编号
    \item 任务(任务编号,任务名称,任务描述)
    \item 数据集(数据集编号,数据集名称,数据集描述,数据集链接,创建日期)
    \item 论文作者(论文编号,作者编号,作者顺序,是否通讯作者)。外键:论文编号,作者编号
    \item 论文方法(论文编号,方法编号)。外键:论文编号,方法编号
    \item  论文评价(论文编号,指标编号,得分,模型实现细节)。外键:论文编号,指标编号
\end{enumerate}

\section{基本表设计}

基本表设计如表\ref{tab:paper} $\sim$ \ref{tab:paper-benchmark} 所示。

\begin{table}[htbp!]
    \centering
    \caption{paper表}
    \begin{tabular}{|l|l|l|l|l|}
    \hline
        属性名 & 数据类型 & 是否可空 & 键 & 解释 \\ \hline
        paperId & int & 否 & 主键 & 论文编号 \\ \hline
        title & varchar(255) & 否 &  & 论文标题 \\ \hline
        paperLink & varchar(255) & 否 &  & 论文链接 \\ \hline
        abs & varchar(255) & 否 &  & 简短摘要 \\ \hline
        publishDate & date & 否 &  & 发表日期 \\ \hline
    \end{tabular}
    \label{tab:paper}
\end{table}

\begin{table}[htbp!]
    \centering
    \caption{author表}
    \begin{tabular}{|l|l|l|l|l|}
    \hline
        属性名 & 数据类型 & 是否可空 & 键 & 解释 \\ \hline
        authorId & int & 否 & 主键 & 作者编号 \\ \hline
        authorName & varchar(255) & 否 &  & 姓名 \\ \hline
        inst & varchar(255) & 是 &  & 机构 \\ \hline
    \end{tabular}
    \label{tab:author}
\end{table}

\begin{table}[htbp!]
    \centering
    \caption{code表}
    \begin{tabular}{|l|l|l|l|l|}
    \hline
        属性名 & 数据类型 & 是否可空 & 键 & 解释 \\ \hline
        codeId & int & 否 & 主键 & 代码编号 \\ \hline
        paperId & int & 否 & 外键 & 论文编号 \\ \hline
        codeLink & varchar(255) & 否 &  & 代码链接 \\ \hline
        stars & int & 否 &  & 收藏数 \\ \hline
        framework & varchar(255) & 否 &  & 框架 \\ \hline
    \end{tabular}
    \label{tab:code}
\end{table}

\begin{table}[htbp!]
    \centering
    \caption{method表}
    \begin{tabular}{|l|l|l|l|l|}
    \hline
        属性名 & 数据类型 & 是否可空 & 键 & 解释 \\ \hline
        methodId & int & 否 & 主键 & 方法编号 \\ \hline
        methodName & varchar(255) & 否 &  & 方法名 \\ \hline
        methodDesc & varchar(255) & 是 &  & 方法描述 \\ \hline
    \end{tabular}
    \label{tab:method}
\end{table}

\begin{table}[htbp!]
    \centering
    \caption{benchmark表}
    \begin{tabular}{|l|l|l|l|l|}
    \hline
        属性名 & 数据类型 & 是否可空 & 键 & 解释 \\ \hline
        benchId & int & 否 & 主键 & 指标编号 \\ \hline
        metric & varchar(255) & 否 &  & 得分计算方式 \\ \hline
        datasetId & int & 否 & 外键 & 数据集编号 \\ \hline
        taskId & int & 否 & 外键 & 任务编号 \\ \hline
    \end{tabular}
    \label{tab:benchmark}
\end{table}

\begin{table}[htbp!]
    \centering
    \caption{task表}
    \begin{tabular}{|l|l|l|l|l|}
    \hline
        属性名 & 数据类型 & 是否可空 & 键 & 解释 \\ \hline
        taskId & int & 否 & 主键 & 任务编号 \\ \hline
        taskName & varchar(255) & 否 &  & 任务名称 \\ \hline
        taskDesc & varchar(255) & 是 &  & 任务描述 \\ \hline
    \end{tabular}
    \label{tab:task}
\end{table}

\begin{table}[htbp!]
    \centering
    \caption{dataset表}
    \begin{tabular}{|l|l|l|l|l|}
    \hline
        属性名 & 数据类型 & 是否可空 & 键 & 解释 \\ \hline
        datasetId & int & 否 & 主键 & 数据集编号 \\ \hline
        datasetName & varchar(255) & 否 &  & 数据集名称 \\ \hline
        datasetDesc & varchar(255) & 是 &  & 数据集描述 \\ \hline
        datasetLink & varchar(255) & 否 &  & 数据集链接 \\ \hline
        createDate & date & 否 &  & 创建日期 \\ \hline
    \end{tabular}
    \label{tab:dataset}
\end{table}

\begin{table}[htbp!]
    \centering
    \caption{paper-author表}
    \begin{tabular}{|l|l|l|l|l|}
    \hline
        属性名 & 数据类型 & 是否可空 & 键 & 解释 \\ \hline
        paperId & int & 否 & 主键,外键 & 论文编号 \\ \hline
        authorId & int & 否 & 主键,外键 & 作者编号 \\ \hline
        ord & int & 否 &  & 作者顺序 \\ \hline
        iscorr & int & 否 &  & 是否通讯作者 \\ \hline
    \end{tabular}
    \label{tab:paper-author}
\end{table}

\begin{table}[htbp!]
    \centering
    \caption{paper-method表}
    \begin{tabular}{|l|l|l|l|l|}
    \hline
        属性名 & 数据类型 & 是否可空 & 键 & 解释 \\ \hline
        paperId & int & 否 & 主键,外键 & 论文编号 \\ \hline
        methodId & int & 否 & 主键,外键 & 方法编号 \\ \hline
    \end{tabular}
    \label{tab:paper-method}
\end{table}

\begin{table}[htbp!]
    \centering
    \caption{paper-benchmark表}
    \begin{tabular}{|l|l|l|l|l|}
    \hline
        属性名 & 数据类型 & 是否可空 & 键 & 解释 \\ \hline
        paperId & int & 否 & 主键,外键 & 论文编号 \\ \hline
        benchId & int & 否 & 主键,外键 & 指标编号 \\ \hline
        score & decimal(3,3) & 否 &  & 得分 \\ \hline
        modelDesc & varchar(255) & 是 &  & 模型实现细节 \\ \hline
    \end{tabular}
    \label{tab:paper-benchmark}
\end{table}

