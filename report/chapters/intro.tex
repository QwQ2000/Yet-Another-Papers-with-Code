\chapter{系统需求分析}
\section{系统描述}
机器学习是一种将分析模型构建自动化的数据分析方法。它是人工智能的一个分支,其依据是系统可以从数据中学习、识别模式并以最少的人工干预做出决策。机器学习正在越来越多的领域中得到应用,第四次工业革命正在暗潮涌动。

为了更好的推动不同机器学习算法之间的交流、比较、融合、创新,需要一个公共的、开放的数据平台,为所有研究人员提供最新、最好的论文和算法。

\textbf{Yet Another Papers with Code} 是一个免费开放资源库,收录了大量机器学习论文、代码和及其评估。该系统的核心功能是汇总和整理重要的论文及其对应的代码实现。机器学习论文数量庞大,因此学术界对此已经有了相关评价指标(benchmark),用以衡量论文和算法的优劣程度。评价指标包含其所采用的数据集和所做的机器学习任务(如聚类、深度估计等)。

用户可以从多角度筛选数据。系统中的数据全部由管理员维护,用户只能筛选和查看,不能修改系统中的数据,以保证数据的质量和权威性。


\section{数据存储需求}
本系统中的实体包含:论文、作者、代码、方法、评价指标、数据集、任务。

论文信息包含:论文标题、论文链接、简短摘要、发表时间。

代码信息包含:代码链接、收藏数、所用框架。

评价指标信息包含:得分计算方式、对应的数据集、对应的任务。

数据集信息包含:数据集名称、描述、链接、创建日期。

作者信息包含:姓名、机构。

方法和任务包含:名称、描述。

每一篇论文都附有一份或多份代码实现,而一份代码唯一属于某一篇论文。一篇论文可能涉及多种机器学习方法,可能包含多名作者。

一篇论文可以参与一个或多个评价指标,获得评分并与其他参与该评测的论文共同排名。一个评价指标有唯一的数据集和任务,而任务和数据集可以对应多个评价指标。

作者和论文的关系中包含:作者在该论文中的作者顺序、是否是该论文的通讯作者

论文和评价的关系中包含:论文在该评价指标中的得分、参与评价的模型的实现细节

\section{应用程序功能}
前台主要功能如下(按页面标题罗列):

\begin{itemize}
    \item 首页
    \begin{itemize}
        \item 根据名称模糊查找论文
        \item 根据时间、代码收藏数对论文进行排序
    \end{itemize}
    \item 看SOTA:根据任务对应的论文数量对任务排序
    \item 数据集:
    \begin{itemize}
        \item 根据名称模糊查找数据集
        \item 查询应用到某个任务的数据集
        \item 对上述查询结果按相关论文数或者发布时间排序
    \end{itemize}
    \item 论文页:查看论文详细信息
    \item 方法页:
    \begin{itemize}
        \item 根据方法对应的论文数量对方法排序
        \item 筛选某一方法的所有论文。在该筛选结果中按论文名模糊查找
    \end{itemize}
    \item 任务页:
    \begin{itemize}
        \item 根据任务查询所有相关的评价指标,以及参与该评价指标中得分最高的论文
        \item 根据任务查询所有相关的数据集
    \end{itemize}
    \item 评价指标页:列出某一评价指标的所有相关论文,根据数值列出排行榜
    \item 数据集页:
    \begin{itemize}
        \item 查询某个数据集相关的所有评价指标
        \item 筛选某一数据集的所有论文。在该筛选结果中按论文名模糊查找
    \end{itemize}
    \item 作者页:查找某一作者的所有论文
\end{itemize}
